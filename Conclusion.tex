\section{Conclusion}
\seclab{conclusion}

\subsection{Future Work}
\begin{itemize}
\item \textit{Confidence Networks}: Relax the constraints of proxy voting to allow voters to proxy to multiple people. Then, use centrality measures, not as a means of breaking cycles, but as a way to identify experts in a group.

\item \textit{Behavioral Model of Voting}: What factors lead people to proxy? Quantify various aspects such as confidence, trust, bias, and knowledge and generate functions that say to whom each voter should proxy.
\end{itemize}

In this paper, we explored the idea of using centrality measures to break cycles, our reasoning being that a proxy to someone is an implicit vote of confidence in that person's expertise. If we view the graph of proxy votes as a graph where directed edges represent votes of confidence, then centrality is the natural choice to choose the experts in the graph. However, centrality measures do not require that nodes only "proxy" to one other node --- having this as a requirement in proxy voting limits the information we can obtain from voters, as it forces the graph to be unnecessarily sparse. One potential way to get around this problem is through the use of \textit{confidence networks}, where voters are allowed to list not just one but multiple other voters in which they feel confident. In this case, centrality measures would be used, not as a way to deal with the unfortunate case of cycles, but rather as way to calculate the relative expertise of voters in a system. There are many different ways this could be used --- for instance, votes could be weighted by the calculated expertise of the voters who cast them.

\subsection{Impact}
In this paper, we explained the motivation behind proxy voting and explored the aspect of cycles --- why they form and how to break them. Ultimately, in order to implement a proxy voting system, we need a way to handle cycles. Until now, the only method proposed to deal with cycles involved granting them zero voting-power. By proposing a cycle-breaking method that does not disenfranchise voters in a cycle, we provide allow for the use of proxy voting in situations where voters would be upset if their votes were discarded, such as modern political elections.