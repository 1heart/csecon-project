%%BeginIpePreamble
\newcommand{\pnt}{p}
\newcommand{\query}{q}
\newcommand{\prob}{\alpha}
\newcommand{\val}{v}
\newcommand{\notLPr}{\beta}
\newcommand{\notPr}{\pi}
\newcommand{\probMV}{\gamma}
\newcommand{\exM}{\mu}
\newcommand{\mlM}{\lambda}
\newcommand{\range}{\rho}
\newcommand{\rect}{\range}
\newcommand{\rectangle}{\rho}
\newcommand{\appr}{\eta}
\newcommand{\PNT}{P}
\newcommand{\VALS}{V}
\newcommand{\maxm}{f}


%\renewcommand{\Re}{{\rm I\!\hspace{-0.025em} R}}
\renewcommand{\Re}{\mathbb{R}}

\newcommand{\PntSet}{\mathsf{P}}
\newcommand{\Sample}{\mathsf{R}}
\newcommand{\LocSample}{\mathsf{Q}}
\newcommand{\Prob}[1]{\mathbf{Pr\big[} #1 \mathbf{\big]}}
\newcommand{\st}{\text{ s.t. }}
\newcommand{\brc}[1]{\{ #1 \}}
\newcommand{\CH}[1]{\mathcal{CH}(#1)}
\newcommand{\IndicatorX}[1]{\mathbb{I}[ #1 ]}
\newcommand{\cardin}[1]{| #1 | }
\newcommand{\floor}[1]{\left \lfloor #1 \right \rfloor}
\newcommand{\ceiling}[1]{\left \lceil #1 \right \rceil}
\newcommand{\Interior}[1]{\mathrm{relint}(#1)}
\newcommand{\eps}{\varepsilon}
\newcommand{\Expec}[1]{\mathbf{E}\left [ #1 \right ]}

\newcommand{\polylog}{\mathop {\mathrm{polylog}}}

%\newcommand{\polylog}{\mathrm{polylog}}
\newcommand{\argmax}{\arg\!\max}
%%EndIpePreamble
\newcommand{\seclab}[1]{\label{section:#1}}
\newcommand{\secref}[1]{Section~\ref{section:#1}}


\newcommand{\apndxlab}[1]{\label{appendix:#1}}
\newcommand{\apndxref}[1]{Appendix~\ref{appendix:#1}}

\newcommand{\thmlab}[1]{\label{thm:#1}}
\newcommand{\thmref}[1]{Theorem~\ref{thm:#1}}

\newcommand{\lemlab}[1]{\label{lemma:#1}}
\newcommand{\lemref}[1]{Lemma~\ref{lemma:#1}}

\newcommand{\corlab}[1]{\label{corollary:#1}}
\newcommand{\corref}[1]{Corollary~\ref{corollary:#1}}

\newcommand{\itemlab}[1]{\label{item:#1}}
\newcommand{\itemref}[1]{item~\ref{item:#1}}


\newcommand{\figlab}[1]{\label{fig:#1}}
\newcommand{\figref}[1]{Figure~\ref{fig:#1}}

\newcommand{\algolab}[1]{\label{algorithm:#1}}
\newcommand{\algoref}[1]{Algorithm~\ref{algorithm:#1}}


\newcommand{\Eqlab}[1]{\label{equation:#1}}
\newcommand{\Eqref}[1]{Equation~\ref{equation:#1}}


\newcommand{\conjlab}[1]{\label{conjecture:#1}}
\newcommand{\conjref}[1]{Conjecture~\ref{conjecture:#1}}

\newcommand{\problab}[1]{\label{problem:#1}}
\newcommand{\probref}[1]{Problem~\ref{problem:#1}}

\newtheorem{problem}{Problem}[section]
\newtheorem{theorem}{Theorem}[section]
\newtheorem{lemma}[theorem]{Lemma}
\newtheorem{corollary}[theorem]{Corollary}
%\newenvironment{proof}{\trivlist\item[]\emph{Proof}:}%
%                  {\unskip\nobreak\hskip 1em plus 1fil\nobreak%
%                           \rule{2mm}{2mm}%$\Box$
%                           \parfillskip=0pt%
%                           \endtrivlist}

\newtheorem{remark}{Remark}[theorem]
\theoremstyle{remark}{\theorembodyfont{\rm}

\newtheorem{conjecture}{Conjecture}[theorem]


%\newcommand{\qed}{\nobreak \ifvmode \relax \else
%      \ifdim\lastskip<1.5em \hskip-\lastskip
%      \hskip1.5em plus0em minus0.5em \fi \nobreak
%      \vrule height0.75em width0.5em depth0.25em\fi}

\newcommand{\EM}{\textsf{EM}\xspace}
\newcommand{\MLM}{\textsf{MLM}\xspace}
\newcommand{\remove}[1]{}
\def\etal{\textsl{et~al.}}

\makeatletter
\long\def\@makecaption#1#2{
   \vskip 10pt
   \setbox\@tempboxa\hbox{{\footnotesize \textbf{#1.} #2}}
   \ifdim \wd\@tempboxa >\hsize         % IF longer than one line:
       {\footnotesize \textbf{#1.} #2\par}% THEN set as ordinary paragraph.
     \else                              %   ELSE  center.
       \hbox to\hsize{\hfil\box\@tempboxa\hfil}
   \fi}
\makeatother

