\documentclass[10pt]{article}

\usepackage{amssymb,amsmath,amsthm,amsfonts}

\newtheorem{theorem}{Theorem}[section]
\newtheorem{lemma}[theorem]{Lemma}
\newtheorem{proposition}[theorem]{Proposition}
\newtheorem{corollary}[theorem]{Corollary}

  \begin{document}
  
\title{Cycle Breaking in Proxy Voting
}
\date{May 1, 2016}

\author{
Ben Chesnut, Jeremy Fox, Yixin Lin \thanks{%
Vince Conitzer, Department of Computer Science, Duke University, Durham, NC
27708-0129, USA;}}

\maketitle

\section{Introduction}

\subsection{Introduction to Proxy Voting}

\subsection{Motivation}

\begin{lemma}
Here is an example Lemma.
\end{lemma}
\begin{proof}
The proof is trivial.
\end{proof}
\cite{test}

\section{Previous Work}


\subsection{Pseudoforests}


\section{Our Work}

\subsection{Existence of cycles}

There are cases where we can guarantee no cycles, even reasonably common cases.

Let us define a competence measure $C_i \in \mathbb{R}$ for $ i \in V$, which is the objective measure of voter $i$'s competence in voting. Let the perceived competence measure of voter $j$ perceived by voter $i$ be $c_{ij}$ for $i, j \in V$ . 

At the ideal limit where actual competence equals perceived competence, we have the following monotonicity theorem:

\begin{theorem}

Assume that $i$ will only proxy its vote to $j$ iff $c_{ij} > c_{ii}$, i.e. a voter proxies to a target only if the target has a greater perceived competence than the voter.

If the actual competence equals the perceived competence, i.e. $C_i = c_{ji}$ $\forall j, i \in V$, then there cannot be cycles in this graph.

\end{theorem}

\begin{proof}

Proof by contradiction:

Suppose there exists $Ci$ for $i \in \{1, 2, .. n\}$ for which nodes $i$ form a cycle $1 \rightarrow 2 \rightarrow ... n \rightarrow 1$. Then $C_1 < C_2 < ... C_n < C_1$. But the real numbers are totally ordered and therefore transitive, so this is impossible.

\end{proof}

\subsection{Cycle Formation}

\begin{theorem}
CFLIT!!!!
\end{theorem}
\begin{proof}
The proof is left as an exercise for the reader.
\end{proof}

\subsection{Centrality}
\section{Conclusion}

\subsection{Future Work}
\begin{itemize}
\item \textit{Confidence Networks}: Relax the constraints of proxy voting to allow voters to proxy to multiple people. Then, use centrality measures, not as a means of breaking cycles, but as a way to identify experts in a group.

\item \textit{Behavioral Model of Voting}: What factors lead people to proxy? Quantify various aspects such as confidence, trust, bias, and knowledge and generate functions that say to whom each voter should proxy.
\end{itemize}

In this paper, we explored the idea of using centrality measures to break cycles, our reasoning being that a proxy to someone is an implicit vote of confidence in that person's expertise. If we view the graph of proxy votes as a graph where directed edges represent votes of confidence, then centrality is the natural choice to choose the experts in the graph. However, centrality measures do not require that nodes only "proxy" to one other node --- having this as a requirement in proxy voting limits the information we can obtain from voters, as it forces the graph to be unnecessarily sparse. One potential way to get around this problem is through the use of \textit{confidence networks}, where voters are allowed to list not just one but multiple other voters in which they feel confident. In this case, centrality measures would be used, not as a way to deal with the unfortunate case of cycles, but rather as way to calculate the relative expertise of voters in a system. There are many different ways this could be used --- for instance, votes could be weighted by the calculated expertise of the voters who cast them.

\subsection{Impact}
In this paper, we explained the motivation behind proxy voting and explored the aspect of cycles --- why they form and how to break them. Ultimately, in order to implement a proxy voting system, we need a way to handle cycles. Until now, the only method proposed to deal with cycles involved granting them zero voting-power. By proposing a cycle-breaking method that does not disenfranchise voters in a cycle, we provide allow for the use of proxy voting in situations where voters would be upset if their votes were discarded, such as modern political elections.

\begin{thebibliography}{9}
\bibitem{preferential} 
Jan Behrens and Bjorn Swierczek
\textit{Preferential Delegation and the Problem of Negative Voting Weight}. 
Berlin, January 23, 2015.
\\\texttt{http://www.liquid-democracy-journal.org/}

\end{thebibliography}

\end{document}