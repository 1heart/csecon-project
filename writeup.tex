\documentclass[10pt]{article}

\usepackage{amssymb,amsmath,amsthm,amsfonts}
\usepackage{mathtools}

\newtheorem{theorem}{Theorem}[section]
\newtheorem{lemma}[theorem]{Lemma}
\newtheorem{proposition}[theorem]{Proposition}
\newtheorem{corollary}[theorem]{Corollary}

\theoremstyle{definition}
\newtheorem{definition}{Definition}[section]

\newcommand{\bb}[1]{\mathbb{#1}}

  \begin{document}
  
\title{Cycle Breaking in Proxy Voting
}
\date{May 1, 2016}

\author{
Ben Chesnut, Jeremy Fox, Yixin Lin \thanks{%
Vince Conitzer, Department of Computer Science, Duke University, Durham, NC
27708-0129, USA;}}

\maketitle

\section{Introduction}

\subsection{Introduction to Proxy Voting}

\subsection{Motivation}

\begin{lemma}
Here is an example Lemma.
\end{lemma}
\begin{proof}
The proof is trivial.
\end{proof}
\cite{test}

\section{Previous Work}


\subsection{Pseudoforests}


\section{Our Work}

We start with some notation. Let $a \rightarrow b$ denote that voter $a$ proxies to voter $b$. We let $\bb{O}$ denote a total order over the set of all voters. Let $G = (V,E)$ denote the graph of proxy votes, where $V$ is the set of all voters, and $E$ is the set of all edges.

\subsection{Formalization as Graph Theory}
To visualize and analyze a proxy voting system, we can represent the system as a graph, where each voter is a node having weight 1 and each proxy vote is a directed edge pointing from the proxying voter to the voter who is being proxied to. As such, voting power flows between nodes along edge paths; each voter has voting power equal to its own weight of 1, plus the sum of the voting power of the nodes pointing to it. We can formally define the voting power $p(v)$ of voter $v$ recursively:

\theoremstyle{definition}
\begin{definition}{Voting Power of a voter:}
For a voter $v, \quad p(v) = 1 + \smashoperator{\sum\limits_{\{i \in V \mid (i,v) \in E\}}}p(i)$
\end{definition}

Furthermore, we can define the voting power $P(Q)$ of a set $Q$ as follows:

\theoremstyle{definition}
\begin{definition}{Voting Power of a set:}
For $Q \subseteq V, P(Q)$ is the total number of votes cast from $Q$.
\end{definition}



Proxy voting graphs also have a few special properties. Particularly, each node in the graph can have at most one outgoing edge, or outdegree at most one. Graphs with this property are called \textbf{directed pseudoforests}. As a consequence, we know that proxy voting graphs are made up of connected components each of which have at most one cycle.




\subsection{Existence of cycles}

There are cases where we can guarantee no cycles, even reasonably common cases.

Let us define a competence measure $C_i \in \mathbb{R}$ for $ i \in V$, which is the objective measure of voter $i$'s competence in voting. Let the perceived competence measure of voter $j$ perceived by voter $i$ be $c_{ij}$ for $i, j \in V$ . 

We assume throughout this paper that $i \rightarrow j \implies c_{ii} < c_{ij}$, i.e., voter $i$ proxies to voter $j$ only if $i$ perceives $j$ to have greater competence than itself.


At the ideal limit where actual competence equals perceived competence, we have the following monotonicity theorem:

\begin{theorem}

If $C_i = c_{ji}$ $\forall j, i \in V$, then $G$ contains no cycles.

\end{theorem}

\begin{proof}

We prove by contradiction

Suppose there exists $C_i$ for $i \in \{1, 2, ..., n\}$ for which nodes $i$ form a cycle $1 \rightarrow 2 \rightarrow ... n \rightarrow 1$. Then $c_{12} < c_{23 }< ... < c_{(n-1)n} < c_{n1} \implies C_1 < C_2 < ... < C_n < C_1 \implies C_1 < C_1. \Rightarrow\Leftarrow$
\end{proof}

\subsection{Cycle Formation}
Let $\bb{S}$ denote the set of all possible proxy voting systems. Let $S \in \bb{S}$ denote an arbitrary proxy voting system.


\theoremstyle{definition}
\begin{definition}{Choice of Role:}
$S$ exhibits Choice of Role if all voters in $S$ can choose whether they vote or proxy.
\end{definition}

\theoremstyle{definition}
\begin{definition}{Enfranchisement:}
$S$ exhibits Enfranchisement if $\forall a \in V, P(V) > P(V\setminus\{a\})$
\end{definition}

\theoremstyle{definition}
\begin{definition}{Voters Choose:}
For a voter $v$, if $v$ voted, let $A_v$ denote the set of candidates for which $v$ voted. If $v$ did not vote, let $A_v$ denote the set of candidates for which $v$ would vote if $v$ were to vote. Let $c$ denote the candidate chosen by the voting rule in $S$. $S$ exhibits Voters Choose if $c \in \smashoperator{\bigcup\limits_{v \in V}}A_v$ 
\end{definition}

\begin{lemma}[Cycles Form]
\label{cyclesform}
If no voter votes, then there must exist some cycle.
\end{lemma}
\begin{proof}
We prove by contradiction:
Suppose that $G$ does not contain a cycle. Then $G$ must consist only of trees, which implies $|E| \leq |V| -1$. However, since nobody voted, everybody proxied, which implies that $|E| = |V|. \Rightarrow\Leftarrow$
\end{proof}

\begin{theorem} 
There exists no system $S$ for which the properties Choice of Role, Enfranchisement, and Voters Choose simultaneously hold.
\end{theorem}
\begin{proof}
Let us assume that all voters proxy, which is allowed under Choice of Rule. We have from Lemma \ref{cyclesform} that $G$ must contain a cycle. 
\end{proof}

\subsection{Centrality}
\section{Conclusion}

\subsection{Future Work}
\begin{itemize}
\item \textit{Confidence Networks}: Relax the constraints of proxy voting to allow voters to proxy to multiple people. Then, use centrality measures, not as a means of breaking cycles, but as a way to identify experts in a group.

\item \textit{Behavioral Model of Voting}: What factors lead people to proxy? Quantify various aspects such as confidence, trust, bias, and knowledge and generate functions that say to whom each voter should proxy.
\end{itemize}

In this paper, we explored the idea of using centrality measures to break cycles, our reasoning being that a proxy to someone is an implicit vote of confidence in that person's expertise. If we view the graph of proxy votes as a graph where directed edges represent votes of confidence, then centrality is the natural choice to choose the experts in the graph. However, centrality measures do not require that nodes only "proxy" to one other node --- having this as a requirement in proxy voting limits the information we can obtain from voters, as it forces the graph to be unnecessarily sparse. One potential way to get around this problem is through the use of \textit{confidence networks}, where voters are allowed to list not just one but multiple other voters in which they feel confident. In this case, centrality measures would be used, not as a way to deal with the unfortunate case of cycles, but rather as way to calculate the relative expertise of voters in a system. There are many different ways this could be used --- for instance, votes could be weighted by the calculated expertise of the voters who cast them.

\subsection{Impact}
In this paper, we explained the motivation behind proxy voting and explored the aspect of cycles --- why they form and how to break them. Ultimately, in order to implement a proxy voting system, we need a way to handle cycles. Until now, the only method proposed to deal with cycles involved granting them zero voting-power. By proposing a cycle-breaking method that does not disenfranchise voters in a cycle, we provide allow for the use of proxy voting in situations where voters would be upset if their votes were discarded, such as modern political elections.

\begin{thebibliography}{9}
\bibitem{preferential} 
Jan Behrens and Bjorn Swierczek
\textit{Preferential Delegation and the Problem of Negative Voting Weight}. 
Berlin, January 23, 2015.
\\\texttt{http://www.liquid-democracy-journal.org/}

\end{thebibliography}

\end{document}